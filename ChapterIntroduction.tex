\renewcommand{\thechapter}{1}
\chapter{Introduction}
\label{ch:Introduction}

The EXO-200 detector is designed to search for a theorized decay mode of xenon-136 by emission of two electrons and no neutrinos.  Neutrinoless double-beta ($\beta\beta 0\nu$) decay, if it occurs, would have far-reaching implications: it would set the absolute mass scale of the neutrino sector, provide clues to the mechanism which generates mass in the neutrino sector, and give the first direct observation of non-conservation of lepton number.

EXO-200 contains roughly 110 kg of liquid xenon in its active volume.  The xenon used by EXO has been enriched to $80.6\%$ in xenon-136, resulting in approximately $3.9 \times 10^{26}$ atoms of $^{136}$Xe which can be observed in the EXO-200 detector.  The observation of $\beta\beta 0\nu$ decay would present itself as a peak in the energy spectrum at a Q-value of $2458$ keV, so EXO-200 is designed to produce a peak at this energy which is subject to as little background as possible.

Background is controlled primarily by reducing the level of radioactivity of materials in and around the liquid xenon.  All materials used in the construction of the EXO-200 detector were carefully screened for low concentration of radioactive backgrounds, making the EXO-200 detector one of the lowest-background detectors in the world.

After construction, background reduction must be performed by discrimination between $\beta\beta 0\nu$ and other processes.  This can be performed primarily by exploiting the event topology and by refining the energy measurements.  The event topology of $\beta\beta 0\nu$ consists of a highly localized energy deposit due to the short interaction length of $\beta$ particles; by contrast, backgrounds will often produce diffuse energy deposits which can be rejected as candidate $\beta\beta 0\nu$ events.

Perhaps the most straightforward method of background reduction, however, is by energy.  An improved energy resolution assures us that $\beta\beta 0\nu$ candidate events must come from a narrower region of the energy spectrum and lets us reject events outside of that region of interest (ROI).  The EXO-200 detector observes energy in two forms, ionization (charge) and scintillation (light), and both are necessary to achieve axcellent energy resolution.  The scintillation measurement has a lower accuracy, so it is the limiting factor to the energy resolution of EXO-200.

In this work, we describe a new technique for improving the accuracy of the scintillation measurements of the EXO-200 detector through a detailed offline waveform analysis.  This technique, identified throughout this work as ``denoising'', will consist of understanding the signal-to-noise ratio of the different components of our scintillation measurement.  The signal-to-noise content of our scintillation measurements will be seen to depend on both the proximity of a light sensor to the energy deposit and on the spectral shape of light pulses compared to waveform noise.  An active noise reduction program is included in this effort, which in turn improves the signal-to-noise content of the scintillation measurements.  An overall improvement in the EXO-200 energy resolution of more that $20\%$ is achieved at the Q-value.

We will provide an overview of the theoretical motivations for the search for $\beta\beta 0\nu$ decay in chapter~\ref{ch:BB0NTheory}.  Chapter~\ref{ch:EXO200Detector} describes the design of the EXO-200 detector, the backgrounds it expects, and some of the techniques it is capable of using to reduce those backgrounds.  Chapter~\ref{ch:DenoisingTheory} derives the mathematical framework of denoising and some practical considerations of its application; chapters~\ref{ch:NoiseMeasurements} and \ref{ch:Lightmap} describe the measurements of electronic noise and light yield which are critical inputs to denoising.  In chapter~\ref{ch:DenoisingResults} we describe the components and results of the EXO-200 analysis, including comparisons between results with and without denoising in sections~\ref{sec:RotatedEnergyCalibration} and \ref{sec:ResultComparison}.  Conclusions and future outlook are contained in chapter~\ref{ch:Conclusions}.
