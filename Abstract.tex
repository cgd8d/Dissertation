%Abstract Page 

\hbox{\ }

\renewcommand{\baselinestretch}{1}
\small \normalsize

\begin{center}
\large{{ABSTRACT}} 

\vspace{3em} 

\end{center}
\hspace{-.15in}
\begin{tabular}{ll}
Title of dissertation:    & {\large  SIGNAL DENOISING FOR EXO-200 }\\
&                           {\large AND AN IMPROVED LIMIT ON} \\
&                           {\large NEUTRINOLESS DOUBLE-BETA DECAY} \\
\ \\
&                          {\large  Clayton G. Davis, Doctor of Philosophy, 2014} \\
\ \\
Dissertation directed by: & {\large  Professor Carter Hall} \\
&  				{\large	 Department of Physics } \\
\end{tabular}

\vspace{3em}

\renewcommand{\baselinestretch}{2}
\large \normalsize

The EXO-200 detector is designed to measure the spectrum of Xenon-136 double-beta ($\beta\beta$) decay.  This extremely low-rate radioactive process is interesting on its own, but more profound is its potential to reveal non-standard neutrinoless double-beta ($\beta\beta 0\nu$) decay.  $\beta\beta 0\nu$ decay, if it occurs, would have important consequences for our understanding of the neutrino sector of the standard model.  It would demonstrate the type of the neutrino mass term; set the mass scale of the neutrino sector; and demonstrate the first direct observation of lepton number non-conservation.

In the EXO-200 detector, we measure the summed energy of the two emitted electrons.  The summed-electron-energy spectrum of standard $\beta\beta$ decay produces a smooth spectrum due to the "missing" energy of the two emitted neutrinos.  In contrast, the non-standard $\beta\beta 0\nu$ decay has no neutrino products, and emits energy only in the form of electrons.  Since all energy is captured by the electrons, the sum of electron energies will always be $2458$ keV; thus, we can discover this process by looking for a peak in the summed electron energy spectrum at this energy.

The sensitivity of the search depends on two basic factors: accumulated Xenon-136 exposure, and expected background near $2458$ keV from non-$\beta\beta 0\nu$ decays.  The EXO-200 detector was designed to minimize background primarily through its construction.  All materials used in the construction of the EXO-200 detector were carefully screened for low concentration of radioactive backgrounds, making the EXO-200 detector one of the lowest-background detectors in the world.

Backgrounds were also minimized in EXO-200 by ensuring that the energy resolution of the detector would be excellent.  Good energy resolution ensures that all monoenergetic peaks (including both $\beta\beta 0\nu$ and background gamma lines) will be sharp.  When background is integrated over the width of the $\beta\beta 0\nu$ expected peak, we wish for the integrated number of expected counts to be as small as possible; when peaks are sharper, we can expect that integral to be smaller.

When the EXO-200 detector turned on, its scintillation signals observed more electronic noise than had been expected based on design goals.  The present work is concerned with improving the energy resolution of the EXO-200 detector by removing electronic noise by an offline denoising process.  It will be concerned with a detailed characterization of the signal and noise behavior of the EXO-200 scintillation measurements; a novel algorithm for denoising such signals; and results from denoised EXO-200 data will be presented, showing an improved $\beta\beta 0\nu$ limit in Xenon-136.  Since this scheme is still in its infancy, possible future improvements will be described throughout, along with discussion of motivations for these improvements.

