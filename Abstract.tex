%Abstract Page 

\hbox{\ }

\renewcommand{\baselinestretch}{1}
\small \normalsize

\begin{center}
\large{{ABSTRACT}} 

\vspace{3em} 

\end{center}
\hspace{-.15in}
\begin{tabular}{ll}
Title of dissertation:    & {\large A Search for the Neutrinoless Double Beta Decay}\\
&                           {\large of Xenon-136 with Improved} \\
&                           {\large Sensitivity from Waveform Denoising} \\
\ \\
&                          {\large  Clayton G. Davis, Doctor of Philosophy, 2014} \\
\ \\
Dissertation directed by: & {\large  Professor Carter Hall} \\
&  				{\large	 Department of Physics } \\
\end{tabular}

\vspace{3em}

\renewcommand{\baselinestretch}{2}
\large \normalsize

\singlespace{
The EXO-200 detector is designed to search for a theorized decay process of xenon-136 called neutrinoless double-beta ($\beta\beta 0\nu$) decay.  $\beta\beta 0\nu$ decay, if it occurs, would have important consequences for our understanding of the neutrino sector of the standard model.  It would demonstrate the type of the neutrino mass term; set the mass scale of the neutrino sector; and demonstrate the first direct observation of lepton number non-conservation.  The $\beta\beta 0\nu$ decay produces a monoenergetic peak, so one important approach to reducing backgrounds for this search is by improving the energy resolution of the detector.

The present work describes a new analysis technique which improves resolution in the scintillation channel by a combination of waveform denoising and weighting of waveform components based on their expected signal-to-noise ratio; the overall resolution of the detector is improved by better than $20\%$.  Application of this method results in a halflife limit on $\beta\beta 0\nu$ decay in xenon-136 of $T_{1/2} > 1.1 \cdot 10^{25}$ years at $90\%$ confidence.  Further improvements which could impact the energy resolution of EXO-200 are also described, and implications for the planned nEXO experiment are considered.
}
