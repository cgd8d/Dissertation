%Abstract Page 

\hbox{\ }

\renewcommand{\baselinestretch}{1}
\small \normalsize

\begin{center}
\large{{ABSTRACT}} 

\vspace{3em} 

\end{center}
\hspace{-.15in}
\begin{tabular}{ll}
Title of dissertation:    & {\large A Search for the Neutrinoless Double Beta Decay}\\
&                           {\large of Xenon-136 with Improved} \\
&                           {\large Sensitivity from Waveform Denoising} \\
\ \\
&                          {\large  Clayton G. Davis, Doctor of Philosophy, 2014} \\
\ \\
Dissertation directed by: & {\large  Professor Carter Hall} \\
&  				{\large	 Department of Physics } \\
\end{tabular}

\vspace{3em}

\renewcommand{\baselinestretch}{2}
\large \normalsize

\singlespace{
The EXO-200 detector is designed to search for the neutrinoless double beta decay of $^{136}$Xe.  $\beta\beta 0\nu$ decay, if it occurs in nature, would demonstrate the fundamental nature of neutrino mass; set the mass scale of the neutrino sector; and demonstrate lepton number non-conservation.  Since the $\beta\beta 0\nu$ decay produces a monoenergetic peak, the energy resolution of the detector is of fundamental importance for the sensitivity of the experiment.

The present work describes a new analysis technique which improves the energy resolution of EXO-200 through a combination of waveform denoising and weighting of waveform components based on their expected signal-to-noise ratio. With this method, the energy resolution of the detector is improved by $21\%$.  Applying this technique to 99.8 kg*years of exposure collected by EXO-200 between October 5, 2011 and September 1, 2013, we find no statistically significant evidence for the presence of $\beta\beta 0\nu$ in the data. We set a half-life limit $T_{1/2} > 1.1 \cdot 10^{25}$ years at $90\%$ confidence.  We also describe further improvements which could impact the energy resolution of EXO-200, and consider implications for the planned nEXO experiment.
}
